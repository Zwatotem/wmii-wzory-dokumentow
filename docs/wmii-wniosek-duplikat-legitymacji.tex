\documentclass[a4paper,11pt]{article}
\usepackage[utf8]{inputenc}
\usepackage{lmodern}
\usepackage[MeX]{polski}
\usepackage{microtype}
\usepackage{indentfirst}
\usepackage{calc}
\usepackage{amsmath}
\usepackage{multicol}
\usepackage[symbol]{footmisc}
\usepackage{fontspec}

\setmainfont{TeX Gyre Pagella}

\usepackage[
pdftitle={Oświadczenie o utracie legitymacji studenckiej},
colorlinks=true,linkcolor=black,urlcolor=black,citecolor=black]{hyperref}
\urlstyle{same}

\usepackage{geometry}
\geometry{total={210mm,297mm},
left=25mm,right=25mm,%
bindingoffset=0mm, top=25mm,bottom=20mm}

\linespread{1.2}
\pagestyle{empty}

\newcommand{\fillField}[2]{
    $\underset{\text{#1}}{\parbox[t]{#2}{\dotfill}}$
}

\renewcommand{\thefootnote}{\fnsymbol{footnote}}

\begin{document}
\noindent
\fillField{(imię i nazwisko studenta)}{7cm} \hfill Kraków, dnia \fillField{}{2cm} \\\\
\fillField{(nr albumu – rok i kierunek studiów)}{7cm} \\\\

\vskip 1.0cm
\begin{center}
{\Large \textbf{Oświadczenie o utracie legitymacji studenckiej}}
\end{center}
\vskip 0.5cm

% \noindent
Zgodnie z § 7 ust. 3 Rozporządzenia Ministra Nauki i Szkolnictwa Wyższego z dnia 14 września 2011 r. w sprawie przebiegu dokumentacji studiów: \textit{„W przypadku zniszczenia lub utraty legitymacji studenckiej student jest obowiązany do niezwłocznego zawiadomienia uczelni.”} \\
% \vskip 1.0cm

\noindent
\textbf{Oświadczam, że moja legitymacja studencka została zagubiona/skradziona\footnote[1]{niepotrzebne skreślić}.} \\

\noindent
Niniejsze oświadczenie jest podstawą do unieważnienia utraconej legitymacji. \\

\noindent
Proszę o wystawienie duplikatu legitymacji studenckiej, po opłaceniu przeze mnie należnej kwoty.

\vskip 1.2cm
\hspace{\fill} \fillField{(podpis studenta)}{7cm} \hspace{2.0cm}

\end{document}
