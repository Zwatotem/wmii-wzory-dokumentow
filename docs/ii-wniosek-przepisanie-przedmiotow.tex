\documentclass{wmiisubmission}

\usepackage[table,xcdraw]{xcolor}
\usepackage{graphicx}
\usepackage{tabularx}

\usepackage[
pdftitle={Wniosek o przepisanie przedmiotów},
colorlinks=true,linkcolor=black,urlcolor=black,citecolor=black]{hyperref}
\urlstyle{same}

\newcounter{footnotemarknum}
\newcommand{\fnm}{\addtocounter{footnotemarknum}{1}\footnotemark}

\newcommand{\fnt}[1]{
    \addtocounter{footnote}{-\value{footnotemarknum}}
    \addtocounter{footnote}{1}
    \footnotetext{#1}
    \setcounter{footnotemarknum}{0}
}

\newcommand{\courseTable}{
    \centering
    \resizebox{\textwidth}{!}{
        \begin{tabularx}{\textwidth}{|X|c|c|c|c|}
            \hline
            {\small\textbf{Przedmiot zaliczony}}\textsuperscript{1} & {\footnotesize\textbf{Forma zajęć/liczba godzin}}\textsuperscript{2} &
            {\small \bf ECTS} & {\small\textbf{Uzyskane oceny}}\textsuperscript{3} & {\small\textbf{Rok akademicki}}\textsuperscript{4}  \\
            \hline
              &   &   &   &   \\
              &   &   &   &   \\
            \hline
            {\small\textbf{Przedmiot w IIiMK (nazwa)}}\textsuperscript{1} & {\footnotesize\textbf{Forma zajęć/liczba godzin}}\textsuperscript{2} &
            {\small \bf ECTS} & {\small\textbf{Ostateczne oceny}}\textsuperscript{5} & {\small\textbf{Rok akademicki}}\textsuperscript{6}  \\
            \hline
              &   &   &   &   \\
              &   &   &   &   \\
            \hline
        \end{tabularx}
    }

    \vskip 1.0cm
}

\begin{document}
\cracowdate
\studentinfo{}{}{}{Informatyka - studia stacjonarne}{I stopnia/II stopnia}
\studentaddress
\addressee{\piotrniemiec}

\vskip 3.0cm

\requesttitle{Wniosek o przepisanie przedmiotów}

\vskip 0.5cm

Zwracam się z uprzejmą prośbą o przepisanie przedmiotów wymienionych w
załączniku do niniejszego podania i zaliczenie ich do programu studiów na
kierunku Informatyka w roku akademickim 20\fillField{1cm}/20\fillField{1cm}.

\vskip 2.0cm

\studentsignature

\vfill

\decision{Decyzja Kierownika}

\pagebreak

\newgeometry{tmargin=1.4cm, bmargin=1cm, lmargin=1.4cm, rmargin=1.4cm}

\courseTable
\courseTable
\courseTable
\courseTable
\courseTable
\courseTable


\fnt{Faktyczna nazwa przedmiotu}
\fnt{Forma zajęć czyli np. w-wykład, c-ćwiczenia, l-laboratorium, s-seminarium}
\fnt{Oceny uzyskane np. w-5.0 c-5.0}
\fnt{Rok akademicki, w którym przedmiot był faktycznie realizowany}
\fnt{Ostateczna ocena zatwierdzona do wpisania przez Kierownika studiów [pole należy zostawić puste]}
\fnt{Rok akademicki, w którym przedmiot ma być zaliczony}


\end{document}
