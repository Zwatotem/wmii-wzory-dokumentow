\documentclass[a4paper,12pt]{article}
\usepackage[utf8]{inputenc}
\usepackage{lmodern}
\usepackage[MeX]{polski}
\usepackage{microtype}
\usepackage{indentfirst}
\usepackage{calc}
\usepackage{amsmath}
\usepackage[table,xcdraw]{xcolor}
\usepackage{graphicx}
% \usepackage{times}

\usepackage[
pdftitle={Deklaracja tematu pracy dyplomowej},
colorlinks=true,linkcolor=black,urlcolor=black,citecolor=black]{hyperref}
\urlstyle{same}

\usepackage{geometry}
\geometry{total={210mm,297mm},
left=25mm,right=25mm,%
bindingoffset=0mm, top=25mm,bottom=20mm}

\linespread{1.3}
\pagestyle{empty}

\newcommand{\fillField}[2]{
    $\underset{\text{#1}}{\parbox[t]{#2}{\dotfill}}$
}

\renewcommand{\thefootnote}{\fnsymbol{footnote}}

\begin{document}
\noindent
Imię i nazwisko studenta: \fillField{}{6cm} \hfill Kraków, \fillField{}{3cm}\\\\
Nr albumu:\fillField{}{5cm} \\\\
Studia \fillField{}{3cm} stopnia\\

% \vskip 1.0cm

\vskip 2.0cm

\begin{center}
{\Large \underline{\textbf{Deklaracja tematu pracy dyplomowej}}}
\end{center}

\vskip 0.5cm

\noindent
\underline{\textbf{Temat pracy:}} \dotfill\\

\noindent
\null \dotfill \\\\
\underline{\textbf{Temat pracy w języku angielskim:}} \dotfill\\

\noindent
\null \dotfill \\\\
\underline{\textbf{Opiekun:}} \dotfill \\

\vskip 4.0cm

\null\hfill\fillField{(podpis Opiekuna)}{6cm}% \hspace{2.0cm}

\vskip 2.0cm

\null\hfill\fillField{(podpis Studenta)}{6cm}% \hspace{2.0cm}

\vskip 4.0cm


\end{document}

